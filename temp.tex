\documentclass{beamer}
\usepackage{amsmath, amssymb, amsthm, graphicx}
\usetheme{Madrid}

\title{Isomorphism in Union-Closed Sets}
\author{Mohammad Javad Moghaddas Mehr}
\institute{Your Institution}
\date{Cincinnati, March 2025}

\begin{document}

% Title Slide
\begin{frame}
	\titlepage
\end{frame}

% Introduction
\begin{frame}{Introduction & Motivation}
	\begin{itemize}
		\item The \textbf{Union-Closed Sets Conjecture} (Frankl, 1979) states that in any non-empty union-closed family of sets, there exists an element appearing in at least half of the sets.
		\item The conjecture remains open despite significant efforts.
		\item We explore \textbf{isomorphisms} and their structural impact on union-closed families.
	\end{itemize}
\end{frame}

% Background & Historical Context
\begin{frame}{Background & Historical Context}
	\begin{itemize}
		\item The conjecture has been studied for over four decades.
		\item Special cases have been resolved, but a general proof remains elusive.
		\item Connections to extremal combinatorics and algebraic methods.
	\end{itemize}
\end{frame}

% Definitions
\begin{frame}{Definitions}
	\begin{itemize}
		\item A family of sets $\mathcal{K} \subseteq 2^{[n]}$ is \textbf{union-closed} if for any $A, B \in \mathcal{K}$, we have $A \cup B \in \mathcal{K}$.
		\item An \textbf{isomorphism} between two union-closed families $\mathcal{K}_1$ and $\mathcal{K}_2$ is a bijection $h: \mathcal{K}_1 \to \mathcal{K}_2$ such that:
		      \[ h(A \cup B) = h(A) \cup h(B), \quad \forall A, B \in \mathcal{K}_1. \]
	\end{itemize}
\end{frame}

% Example of Union-Closed Family
\begin{frame}{Example: Union-Closed Family}
	\textbf{Consider:}
	\[ \mathcal{K} = \{\emptyset, \{a\}, \{b\}, \{a,b\} \} \]
	\begin{itemize}
		\item Every union of sets in $\mathcal{K}$ remains in $\mathcal{K}$.
		\item Example: $\{a\} \cup \{b\} = \{a, b\} \in \mathcal{K}$.
	\end{itemize}
\end{frame}

% Main Theorem
\begin{frame}{Main Theorem}
	\textbf{Theorem:} (Main Result)
	\begin{itemize}
		\item For every isomorphism $h: \mathcal{K}_1 \to \mathcal{K}_2$, there exists a \textbf{hyperisomorphism} $H: \bigcup \mathcal{K}_1 \to \bigcup \mathcal{K}_2$ such that:
		      \[ h(A) = \{ H(a) \mid a \in A \}, \quad \forall A \in \mathcal{K}_1. \]
	\end{itemize}
\end{frame}

% Proof Sketch
\begin{frame}{Proof Sketch}
	\begin{itemize}
		\item Show that any isomorphism $h$ between union-closed families induces a corresponding bijection $H$ on elements.
		\item Utilize the structure of \textbf{minimal elements} to construct $H$.
		\item Prove that $H$ preserves inclusion and union properties.
	\end{itemize}
\end{frame}

% Step-by-Step Proof (Multiple Slides)
\begin{frame}{Step 1: Establishing the Structure}
	\begin{itemize}
		\item Definition of minimal elements and their role.
		\item Establish basic properties of pure union-closed families.
	\end{itemize}
\end{frame}

\begin{frame}{Step 2: Constructing the Hyperisomorphism}
	\begin{itemize}
		\item Construct $H$ using minimal elements.
		\item Ensure preservation of key properties.
	\end{itemize}
\end{frame}

\begin{frame}{Step 3: Proving Injectivity and Surjectivity}
	\begin{itemize}
		\item Injectivity follows from distinct minimal elements.
		\item Surjectivity follows from properties of isomorphic families.
	\end{itemize}
\end{frame}

\begin{frame}{Step 4: Conclusion of the Proof}
	\begin{itemize}
		\item Verify $H$ satisfies all conditions.
		\item Summarize key insights from the proof.
	\end{itemize}
\end{frame}

% Applications
\begin{frame}{Applications of the Result}
	\begin{itemize}
		\item Provides a structural approach to union-closed families.
		\item May lead to new combinatorial methods for tackling the conjecture.
		\item Opens new research directions in set theory and discrete mathematics.
	\end{itemize}
\end{frame}

% Open Problems
\begin{frame}{Open Problems & Future Work}
	\begin{itemize}
		\item Can hyperisomorphisms be extended to more general set families?
		\item What are the algorithmic implications of these structures?
		\item How can this be applied to the broader context of combinatorial optimization?
	\end{itemize}
\end{frame}

% Conclusion
\begin{frame}{Conclusion}
	\begin{itemize}
		\item We established a connection between \textbf{isomorphisms} and \textbf{hyperisomorphisms} in union-closed families.
		\item This contributes to understanding the structural behavior of these families.
		\item Many open problems remain.
	\end{itemize}
	\bigskip
	\centering \textbf{Thank you!} \quad Questions?
\end{frame}

\end{document}
