\documentclass{beamer}
% Theme choice
\usetheme{metropolis} % Modern and clean layout
\usepackage{amsmath, amssymb, amsthm}
\usepackage{graphicx, fancyhdr}

% Footer settings
\setbeamertemplate{footline}{
  \leavevmode\hbox{\begin{beamercolorbox}[wd=\paperwidth,ht=3ex,dp=2ex]{author in head/foot}%
      \hspace{1em} \insertshorttitle \,
      \textbullet \, 
      \insertsection \, 
      \hfill \insertframenumber{}/\inserttotalframenumber{} \hspace{.5em}
  \end{beamercolorbox}
  }
}

% Title Page Details
\title{Isomorphism in Union-Closed Sets}
\author{Mohammad Javad Moghaddas Mehr}
\date{ORAM 14, Cincinnati, March 2025}

\begin{document}

% Title Slide
\begin{frame}
    \titlepage
\end{frame}

% Outline Slide
\begin{frame}{Outline}
    \tableofcontents
\end{frame}

% Introduction Slide
\section{Introduction}
\begin{frame}{Introduction}
    \begin{itemize}
        \item Definition of Union-Closed Families of Sets
        \item Péter Frankl's Union-Closed Set Conjecture
        \item Our Focus: Structural Properties of Isomorphisms
    \end{itemize}
\end{frame}

% Definitions Slide
\section{Definitions and Theorems}
\begin{frame}{Definitions}
    \textbf{Union-Closed Family:} A collection $\mathcal{K} \subseteq 2^{[n]}$ is union-closed if for all $A, B \in \mathcal{K}$, we have $A \cup B \in \mathcal{K}$.

    \vspace{1em}
    \textbf{Isomorphism:} A bijection $h: \mathcal{K}_1 \to \mathcal{K}_2$ such that:
    \begin{equation*}
        h(A \cup B) = h(A) \cup h(B) \quad \forall A, B \in \mathcal{K}_1.
    \end{equation*}
\end{frame}

% Main Theorem Slide
\begin{frame}{Main Theorem}
    \textbf{Theorem:} For every isomorphism $h: \mathcal{K}_1 \to \mathcal{K}_2$, there exists a corresponding hyperisomorphism $H: \bigcup \mathcal{K}_1 \to \bigcup \mathcal{K}_2$ such that:
    \begin{equation*}
        h(A) = \{H(a) \mid a \in A\}, \quad \forall A \in \mathcal{K}_1.
    \end{equation*}
\end{frame}

% Conclusion Slide
\section{Conclusion}
\begin{frame}{Conclusion}
    \begin{itemize}
        \item Structural preservation under isomorphisms
        \item Connection to the Union-Closed Set Conjecture
        \item Future work: applications of hyperisomorphisms
    \end{itemize}
\end{frame}

% Thank You Slide
\begin{frame}
    \centering
    {\Huge \textbf{Thank You!}}
    \\ Questions?
\end{frame}

\end{document}
