\documentclass{beamer}
\usepackage{amsmath, amssymb, amsthm}
\usepackage{graphicx, fancyhdr}
\usepackage{mdframed} % Add this to the preamble
\usepackage{xcolor} % Ensure xcolor is included
\definecolor{Teal}{HTML}{23373B} % Define the color
% Theme choice
\usetheme{metropolis} % Modern and clean layout


\newtheorem{conjecture}{Conjecture}
% Footer settings
\setbeamertemplate{footline}{
  \leavevmode\hbox{\begin{beamercolorbox}[wd=\paperwidth,ht=3ex,dp=2ex]{author in head/foot}%
      \hspace{1em} {M. J. Moghaddas Mehr}
      \, \textbar \,
      \insertshorttitle \,
      \textbullet \, 
      \insertsection \, 
      \hfill \insertframenumber{}/\inserttotalframenumber{} \hspace{.5em}
  \end{beamercolorbox}
  }
}

% Title Page Details
\title{Isomorphism in Union-Closed Sets}
\author{Mohammad Javad Moghaddas Mehr}
\date{Cincinnati – March 2025}

\begin{document}

% Title Slide
\begin{frame}
    \titlepage
\end{frame}


% Slide 2: History of the Union-Closed Conjecture
\begin{frame}{History}
    In 1979, Péter Frankl proposed a famous conjecture about finite union-closed families. 
    He stated that in every such family, there exists an element that appears in at least half of the sets. 
    Despite significant efforts, the problem has remained unsolved for more than four decades. 
\vspace{0.3cm}
    {\setbeamercolor{block title}{fg=white, bg=Teal!90} % Block title color
     \setbeamercolor{block body}{fg=black, bg=gray!15}  % Block body color
    \begin{block}{Union-Closed Conjecture (Frankl, 1979)}
      Let \(\mathcal{K} \subseteq 2^{[n]}\) be a union-closed family of sets. 
      Then there exists an element \(i \in \bigcup \mathcal{K}\) such that:\(        |\mathcal{K}| \leq 2|\mathcal{K}^i|,\)
      where 
      \[
        \mathcal{K}^i = \{A \in \mathcal{K} \mid i \in A\}.
      \]
    \end{block}}

\end{frame}

\begin{frame}{Example}
      \textbf{Example 1:} Consider the family of sets:
    \[
    \mathcal{K} = \{\emptyset, \{a\}, \{b\}, \{a,b\}\}
    \]
    The union-closed property holds, and element \(a\) (or \(b\)) appears in at least half the sets.


          \textbf{Example 2:} Consider the family of sets:
    \[
    \mathcal{K} = \{\emptyset, \{a\}, \{b\}, \{a,b\}\}
    \]
    The union-closed property holds, and element \(a\) (or \(b\)) appears in at least half the sets.
\end{frame}





\begin{frame}{Thank You!}
    \textbf{Any Questions?}
    \begin{itemize}
        \item Email: m.moghadas11235@gmail.com
        \item Paper available on ArXiv.
    \end{itemize}
\end{frame}

\end{document}
